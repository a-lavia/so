\section{Ejercicio 4}

Para este ejercicio se pedía que completemos la implementación del scheduler Round-Robin implementando los métodos de la clase \textbf{SchedRR}. La consigna dice que esta implementación recibe como parametros la cantidad de núcleos y los valores de sus respectivos \textit{quantums} y que debe utilizar una única cola global.
~
Como el constructor recibe como parametros la cantidad de núcleos y los valores de sus respectivos \textit{quantums}, cargamos dos vectores del tamaño de la cantidad de cores con cada uno de los \textit{quantums} de cada uno.

\begin{itemize}
	\item \textbf{quantum\_por\_core} = Agrego los \textit{quantums} de todos los cores
	\item \textbf{quantum\_proceso} = Lo creo copiando al vector quantum\_por\_core
\end{itemize}

Con esto puedo saber, utilizando al core como índice, cual es el \textit{quantum} total del core y cuanto de ese fue utilizado por el proceso que está ejecutandose en ese core.

~

Luego, poseémos una única cola llamada \textbf{cola\_procesos} en donde vamos encolando los que van llegando y los que están listos luego de estar bloqueados.

~

Las funciones \textbf{void load(int pid)} y \textbf{void unblock(int pid)} encolan al proceso que es pasado por parametro.

~

La función \textbf{int tick(int cpu, const enum Motivo m)} tiene un comportamiento mas particular que pasamos a detallar en el siguiente pseudocódigo:

~

\begin{algorithmic}
\Function{int tick}{int cpu, const enum Motivo m}

	\If {\emph{Si el motivo es que el proceso terminó}}
	\newline
		
		\If {\emph{Si hay encoladas tareas}}

			\State Desencolo un proceso listo de \textbf{cola\_procesos} y lo devuelvo.

		\Else
			\State Devuelvo IDLE\_TASK
		\EndIf

	\State Actualizo el quantum del core aunque no haya tareas.
	\newline
	
	\ElsIf {\emph{Si el motivo es que el proceso se bloqueó}}
		\State Desencolo un proceso listo de \textbf{cola\_procesos} y lo devuelvo.
	\newline
	\Else
		\If{\emph{Si el proceso no es IDLE\_TASK}}
			\State Resto en uno su quantum
			\If{\emph{Si el quantum es cero}}
				\State Llamo a la función \textbf{current\_pid} para saber el pid de la tarea actual.
				\State La encolo en \textbf{cola\_procesos}
				\State Desencolo el proximo proceso de \textbf{cola\_proceso} y lo devuelvo.
			\Else
				\State Como no terminó su quantum llamo a \textbf{current\_pid} y devuelvo el proceso actual.
			\EndIf
		\Else
			
			\If{\emph{Si \textbf{cola\_procesos} tiene algún proceso}}
				\State Lo desencolo y lo devuelvo.
			\EndIf

		\EndIf
	\EndIf
\EndFunction	
\end{algorithmic}
