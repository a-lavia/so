\section{Ejercicio 4}

Para este ejercicio se pedía que completemos la implementación del scheduler Round-Robin implementando los métodos de la clase \textbf{SchedRR}. La consigna dice que esta implementación recibe como parametros la cantidad de núcleos y los valores de sus respectivos \textit{quantums} y que debe utilizar una única cola global.
~
Como el constructor recibe como parametros la cantidad de núcleos y los valores de sus respectivos \textit{quantums}, cargamos dos vectores del tamaño de la cantidad de cores con cada uno de los \textit{quantums} de cada uno.

\begin{itemize}
	\item \textbf{quantum\_por\_core} = Agrego los \textit{quantums} de todos los cores
	\item \textbf{quantum\_proceso} = Lo creo copiando al vector quantum\_por\_core
\end{itemize}

Con esto puedo saber, utilizando al core como índice, cual es el \textit{quantum} total del core y cuanto de ese fue utilizado por el proceso que está ejecutandose en ese core.

~

Luego, poseémos una única cola llamada \textbf{cola\_procesos} en donde vamos encolando los que van llegando y los que están listos luego de estar bloqueados.

~

Las funciones \textbf{void load(int pid)} y \textbf{void unblock(int pid)} encolan al proceso que es pasado por parametro.

~

La función \textbf{int tick(int cpu, const enum Motivo m)} tiene un comportamiento más particular que pasamos a detallar en el siguiente pseudocódigo:

~

\begin{algorithmic}
\Function{int tick}{int cpu, const enum Motivo m}

	\If {\emph{Si la tarea actual es la IDLE\_TASK y si la cola \textbf{cola\_procesos} no esta vacía}}
		\State Actualizo el quantum del core.
		\State Desencolo un proceso de \textbf{cola\_procesos} y lo devuelvo.
			
	\ElsIf{\emph{Si se produjo un nuevo tick de clock}}
		\State Resto en uno al quantum del proceso actual.
		\newline
		\If{\emph{Si el quantum del proceso actual es igual a cero y la cola \textbf{cola\_procesos} no esta vacía}}
			\State Encolo el proceso actual.
			\State Desencolo el proximo de la cola y lo devuelvo
			\State Actualizo el quantum del core.
		\ElsIf{\emph{Si el quantum del proceso actual es igual a cero y la cola \textbf{cola\_procesos} está vacía}}
			\State Devuelvo el proceso que esta corriendo.
			\State Actualizo el quantum del core.
		\EndIf
		\newline
	\ElsIf{\emph{Si el motivo es que el proceso se bloqueo o terminó y la cola \textbf{cola\_procesos} no esta vacía}}
		\State Actualizo el quantum del core.
		\State Desencolo un proceso de \textbf{cola\_procesos} y lo devuelvo.
	\Else
		\State Devuelvo IDLE\_TASK
	
	\EndIf
\EndFunction	
\end{algorithmic}
