\section{Ejercicio 9}

\subsection{Enunciado}
Completar la implementación del scheduler \textit{Multilevel Feedback Queue} implementando los métodos de la clase \textbf{SchedMFQ} en los archivos \textbf{sched\_mfq.cpp} y \textbf{sched\_mfq.h}.
La implementación debe utilizar \textit{n} colas con \textit{Round-Robin} en cada una con los parámetros que se detallan a continuación.

\begin{itemize}
\item Las n colas se numeran de 0 a $n-1$ siendo 0 la de mayor prioridad.

\item El constructor recibe como parámetro $n$ números $q_i$ indicando el \textit{quantum} de la cola $i$.

\item Al iniciar una tarea comienza al final de la cola de mayor prioridad.

\item Siempre se ejecuta la primer tarea de la cola no vacía de mayor prioridad. Si esta tarea consume todo su \textit{quantum} sin bloquearse entonces pasa al final de la cola inmediatamente de inferior prioridad (si hay). Si esta tarea se bloquea antes de agotar su \textit{quantum}, entonces (cuando se desbloquee)

\item Si todas las colas están vacías se ejecuta \textbf{IDLE TASK}.

\end{itemize}

Realice pruebas y muestre su ejecución.

\subsection{Resolución}