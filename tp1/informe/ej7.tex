\section{Ejercicio 7}

Para este ejercicio se pedía realizar las implementaciones de dos nuevos algoritmos de scheduling, del tipo \emph{shortest job first}, que sólo corren tareas del tipo TaskCPU.
La primera de ellas, llamada \textbf{SJF}, elige y ejecuta siempre la tarea con el menor tiempo de CPU que necesite hasta terminar.
La segunda, llamada \textbf{RSJF}, también elige y ejectua la tarea con el menor tiempo de CPU que necesite pero, cada core posée un quantum por lo que cuando se termina éste, se toma una nueva tarea con el menor tiempo de CPU, se encola la tarea desalojada y luego se reinicia el quantum del core.

\subsection{Implementación \textbf{SJF}}

Esta implementación toma como parametros la cantidad de cores y los tiempos de cada proceso a ser ejecutado en el orden en que se los carga.

Para facilitar el manejo de los procesos con sus respectivos tiempos, generamos un struct llamado \emph{Proceso} que tiene como estructura el pid del proceso y su tiempo de ejecución. También posée el operador $<$ que sirve para comparar los tiempos de los procesos.

Con esto, el scheduler posée como estructra dos colas que detallamos a continuación:

\begin{itemize}

\item \emph{tiempos\_procesos}: una cola normal que se utiliza para encolar los tiempos de los procesos que recibe como parametro este scheduler.

\item \emph{cola\_procesos}: una cola de prioridad de la estructura \emph{Proceso}, donde la máxima prioridad la tiene el proceso con menor tiempo de ejecución.

\end{itemize}

El constructor de este scheduler simplemente encola los tiempos de los procesos en la cola \emph{tiempos\_procesos}.

La función \textbf{load} crea un nuevo Proceso con el pid que recibe como parametro y el primer elemento de la cola (ya que ésta tiene los tiempos de los elementos en el orden que van llegando) y encolo este Proceso en la cola de prioridad \emph{cola\_procesos}.

La función \textbf{unblock} no se utiliza ya que, por enunciado, este scheduler sólo corre tareas del tipo TaskCPU (que no tiene llamadas bloqueantes).

La función \textbf{tick} en este scheduler es muy sencilla, pasamos a mostrar un pseudocódigo que muestre su comportamiento:

~

\begin{algorithmic}
\Function{int tick}{int cpu, const enum Motivo m}

	\If {\emph{(Si la tarea actual es la IDLE\_TASK o el motivo es que terminó el proceso actual) y la cola de prioridad \textbf{cola\_procesos} no está vacía}}
		\State Desencolo de \textbf{cola\_procesos} el primero elemento (o sea, el que tiene el menor tiempo) y lo devuelvo.
	
	\ElsIf{\emph{Si el motivo es que pasó un tick de reloj}}
		\State Devuelvo el proceso actual.

	\Else
		\State Devuelvo IDLE\_TASK.
	\EndIf
\EndFunction	
\end{algorithmic}


\subsection{Implementación \textbf{RSJF}}

Esta implementación toma como parametros la cantidad de cores, sus quantums y los tiempos de cada proceso a ser ejecutado en el orden en que se los carga.

Esta implementación también posee un struct llamado \emph{ProcesoRSJF} que es igual al de la implementación anterior.

Este scheduler posée como estructra lo que detallamos a continuación:

\begin{itemize}

\item \emph{quantum\_por\_core}: vector que posee el quantum de cada core.

\item \emph{quantum\_actual}: vector que posee el quantum actual del proceso que esta corriendo en el core.

\item \emph{proceso\_en\_core}: vector de la estructura \emph{ProcesoRSJF}, de tamaño de cantidad de cores, en donde guardo el proceso que actualmente esta corriendo en cada core.

\item \emph{tiempos\_procesos}: una cola normal que se utiliza para encolar los tiempos de los procesos que recibe como parametro este scheduler.

\item \emph{cola\_procesos}: una cola de prioridad de la estructura \emph{ProcesoRSJF}, donde la máxima prioridad la tiene el proceso con menor tiempo de ejecución.

\end{itemize}

En el constructor, iniciamos los vectores \emph{quantum\_por\_core}y \emph{quantum\_actual} con el quantum de cada core y al vector \emph{proceso\_en\_core} con procesos vacios (pid = 0 y tiempo = 0). También encolamos en la cola \emph{tiempos\_procesos} los tiempos de cada proceso.

La función \textbf{load} y \textbf{unblock} son iguales que en la implementación anterior.

La función \textbf{tick} se comporta de la siguiente forma:

~

\begin{algorithmic}
\Function{int tick}{int cpu, const enum Motivo m}

	\If {\emph{(Si la tarea actual es la IDLE\_TASK y la cola de prioridad \textbf{cola\_procesos} no está vacía}}
		\State Desencolo de \textbf{cola\_procesos} el primero elemento (o sea, el que tiene el menor tiempo) y lo devuelvo.
		\State Actualizo la estructura y quantum.
	
	\ElsIf{\emph{Si el motivo es que el proceso terminó y la \textbf{cola\_procesos} no esta vacía}}
		\State Desencolo de \textbf{cola\_procesos} el primero elemento y lo devuelvo.
		\State Actualizo la estructura y quantum.

	\ElsIf{\emph{Si el motivo es que paso un tick.}}
		\State Actualizo el tiempo del proceso que se esta ejecutando y el quantum del core.

		\If{\emph{Si el quantum del core terminó y la \textbf{cola\_procesos} no esta vacía}}
			\State Desencolo de \textbf{cola\_procesos} el primero elemento y lo devuelvo.
			\State Actualizo la estructura y quantum.

		\ElsIf{\emph{Si solo el quantum del core terminó}}
			\State Actualizo el quantum y devuelvo el proceso actual para que siga corriendo.

		\Else
			\State Devuelvo el proceso actual para que siga corriendo.
		\EndIf

	\Else
		\State Devuelvo IDLE\_TASK.
	\EndIf
\EndFunction	
\end{algorithmic}