\section{Introducción}
En este informe presentaremos los algoritmos realizados para la implementación de una \textbf{DHT Kademlia} simplificada, para ello utilizamos la interfaz \textbf{MPI} que nos servirá para pasar mensajes a través de los nodos que componen nuestra red.

\section{Ejercicio 1}

\subsection{\_\_find\_nodes}
Para la implementación de \textbf{\_\_find\_nodes} preguntamos a los nodos mínimos por el hash utilizando \textbf{MPI.COMM\_WORLD.send()} y recibimos la respuesta con \textbf{MPI.COMM\_WORLD.recv()} estas funciones de MPI son bloqueantes, el código es el siguiente:

\begin{lstlisting}
def __find_nodes(self, contact_nodes, thing_hash):

    queue = contact_nodes
    processed = set()
    nodes_min = {}

    nodes_min[self.__hash] = self.__rank

    processed.add(self.__rank)

    while len(queue) > 0:
        node_hash, node_rank = queue.pop()
        if node_rank not in processed:
            processed.add(node_rank)

            self.__comm.send(thing_hash, dest=node_rank
					, tag=TAG_NODE_FIND_NODES_REQ)

            res = self.__comm.recv(source=node_rank
						, tag=TAG_NODE_FIND_NODES_RESP)

            queue.extend(res)
            
            for (nhash, nrank) in res:
                nodes_min[nhash] = nrank

    return nodes_min
\end{lstlisting}


\subsection{\_\_find\_nodes\_join}
Para la implementación de \textbf{\_\_find\_nodes\_join} 

\begin{lstlisting}
def __find_nodes_join(self, contact_nodes):
    nodes_min = set()

    processed = set()
    queue = contact_nodes
    mas_cercanos = {}

    processed.add(self.__rank)

    while len(queue) > 0:
        nhash, nrank = queue.pop()                                 
        if nrank not in processed:
            processed.add(nrank)

            self.__comm.send((self.__hash,self.__rank)
				, dest=nrank, tag=TAG_NODE_FIND_NODES_JOIN_REQ)
            
            res = self.__comm.recv(source=nrank
				, tag=TAG_NODE_FIND_NODES_JOIN_RESP)

            queue.extend(res[0])
            for fhash, fpath in res[1].items():
                self.__files[fhash] = fpath

            for node_hash, node_rank in res[0]:
                if node_rank != self.__rank:
                    mas_cercanos[node_hash] = node_rank


    for n in mas_cercanos.items():
        nodes_min.add(n)

    return nodes_min
\end{lstlisting}


\subsection{\_\_handle\_console\_store}

\begin{lstlisting}
def __handle_console_store(self, data):

    file_hash, file_name = data

    print("[D] [{:02d}] [CONSOLE|STORE] Almacenando archivo '{}' con hash '{}'".format(self.__rank, file_name, file_hash))

    nodes_min_local = self.__get_local_mins(file_hash)

    print("[D] [{:02d}] [CONSOLE|STORE] Buscando nodo más cercano para almacenando archivo '{}' con hash '{}'".format(self.__rank, file_name, file_hash))

    nodes_min = self.__find_nodes(nodes_min_local, file_hash)

    nodes_min = self.__get_mins(nodes_min, file_hash)

    for min_hash, min_rank in nodes_min:
        if self.__rank == min_rank:
            self.__handle_node_store_req(data)
        else:
            self.__comm.send(data, dest=min_rank, tag=TAG_NODE_STORE_REQ)

\end{lstlisting}


\subsection{\_\_handle\_console\_look\_up}

\begin{lstlisting}
def __handle_console_look_up(self, source, data):

    file_hash = data

    print("[D] [{:02d}] [CONSOLE|LOOK-UP] Buscando archivo con hash '{}'".format(self.__rank, file_hash))

    nodes_min_local = self.__get_local_mins(file_hash)

    nodes_min = self.__find_nodes(nodes_min_local, file_hash)

    res = False
    for min_hash, min_rank in nodes_min.items():
        if not res and min_rank != self.__rank:
            self.__comm.send(file_hash, dest=min_rank, tag=TAG_NODE_LOOKUP_REQ)
            res = self.__comm.recv(source=min_rank, tag=TAG_NODE_LOOKUP_RESP)
        
        elif min_rank == self.__rank and file_hash in self.__files:
            res = self.__files[file_hash]


    self.__comm.send(res, dest=source, tag=TAG_CONSOLE_LOOKUP_RESP)
\end{lstlisting}

\section{Ejercicio 2}

\section{Ejercicio 3}
Para la implementación no bloqueante de \textbf{\_\_find\_nodes} y  \textbf{\_\_find\_nodes\_join} reemplazamos las funciones \textbf{MPI.COMM\_WORLD.send()} y \textbf{MPI.COMM\_WORLD.recv()} por la versión no bloqueante de las mismas, estas son \textbf{MPI.COMM\_WORLD.isend()} y \textbf{MPI.COMM\_WORLD.irecv()} esta retornan de inmediato sin bloquearse y luego hacemos uso la función \textbf{MPI.COMM\_WORLD.wait()} que si espera de manera bloqueante el mensaje. Primero enviamos todos los mensajes y luego mediante la función \textbf{MPI.COMM\_WORLD.iprobe()} que indica si hay o no mensajes de manera no bloqueante los vamos recibiendo. El código es el siguiente:

\subsection{\_\_find\_nodes}
\begin{lstlisting}
def __find_nodes(self, contact_nodes, thing_hash):

    queue = contact_nodes
    processed = set()
    nodes_min = {}

    nodes_min[self.__hash] = self.__rank

    processed.add(self.__rank)

    while len(queue) > 0:
        node_hash, node_rank = queue.pop()
        if node_rank not in processed:
            processed.add(node_rank)

            req = self.__comm.isend(thing_hash, dest=node_rank
					, tag=TAG_NODE_FIND_NODES_REQ)

            res = req.wait()

    while self.__comm.iprobe() > 0:
        req = self.__comm.irecv(source=MPI.ANY_SOURCE
				, tag=TAG_NODE_FIND_NODES_RESP)

        res = req.wait()
        queue.extend(res)

        for (nhash, nrank) in res:
            nodes_min[nhash] = nrank


    return nodes_min
\end{lstlisting}

\subsection{\_\_find\_nodes\_join}
\begin{lstlisting}
def __find_nodes_join(self, contact_nodes):

    nodes_min = set()
    processed = set()
    queue = contact_nodes
    mas_cercanos = {}

    processed.add(self.__rank)
                
    while len(queue) > 0:
        node_hash, node_rank = queue.pop()                                 
        if node_rank not in processed:
            processed.add(node_rank)
            req = self.__comm.isend((self.__hash,self.__rank), dest=node_rank
				, tag=TAG_NODE_FIND_NODES_JOIN_REQ)

            req.wait()

    while self.__comm.iprobe() > 0:

        req = self.__comm.irecv(source=MPI.ANY_SOURCE
			, tag=TAG_NODE_FIND_NODES_JOIN_RESP)

        res =  req.wait()
        minimo = res[0]
        files = res[1]

        queue.extend(minimo)

        for fhash, fpath in res[1].items():
            self.__files[fhash] = fpath

        for nhash, nrank in res[0]:
            if nrank != self.__rank:
                mas_cercanos[nhash] = nrank

    for n in mas_cercanos.items():
        nodes_min.add(n)

    return nodes_min
\end{lstlisting}

\section{Ejercicio 4}

