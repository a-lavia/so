\documentclass[10pt, a4paper, spanish]{article}
\usepackage[paper=a4paper, left=1.5cm, right=1.5cm, bottom=1.5cm, top=3.5cm]{geometry}
\usepackage[spanish]{babel}
\selectlanguage{spanish}
\usepackage[utf8]{inputenc}
\usepackage[T1]{fontenc}
\usepackage{indentfirst}
\usepackage{fancyhdr}
\usepackage{latexsym}
\usepackage{lastpage}
\usepackage[colorlinks=true, linkcolor=blue]{hyperref}

\usepackage{listings}
\usepackage{listingsutf8}
\usepackage{color}

\definecolor{codegreen}{rgb}{0,0.6,0}
\definecolor{codegray}{rgb}{0.5,0.5,0.5}
\definecolor{codepurple}{rgb}{0.58,0,0.82}
\definecolor{backcolour}{rgb}{0.95,0.95,0.92}

\lstset{inputencoding=utf8/latin1,
	language=C++,
	basicstyle=\ttfamily,
	keywordstyle=\bfseries\color{blue},
	stringstyle=\color{red}\ttfamily,
	commentstyle=\color{mygreen}\ttfamily,
	morecomment=[l][\color{magenta}]{\#},
	% numbers=left,
	numberstyle=\color{gray},
	backgroundcolor=\color{backcolour},   
	keywordstyle=\color{magenta},
	breakatwhitespace=false,
	breaklines=true,
	captionpos=b,
	keepspaces=true,
	numbersep=5pt,
	showspaces=false,
	showstringspaces=false,
	showtabs=false,
	tabsize=3,
	inputencoding=utf8/latin1
}

% Para que tenga acentos el environment lstlisting
\lstset{
     literate=%
         {á}{{\'a}}1
         {í}{{\'i}}1
         {é}{{\'e}}1
         {ý}{{\'y}}1
         {ú}{{\'u}}1
         {ó}{{\'o}}1
         {ě}{{\v{e}}}1
         {š}{{\v{s}}}1
         {č}{{\v{c}}}1
         {ř}{{\v{r}}}1
         {ž}{{\v{z}}}1
         {ď}{{\v{d}}}1
         {ť}{{\v{t}}}1
         {ň}{{\v{n}}}1                
         {ů}{{\r{u}}}1
         {Á}{{\'A}}1
         {Í}{{\'I}}1
         {É}{{\'E}}1
         {Ý}{{\'Y}}1
         {Ú}{{\'U}}1
         {Ó}{{\'O}}1
         {Ě}{{\v{E}}}1
         {Š}{{\v{S}}}1
         {Č}{{\v{C}}}1
         {Ř}{{\v{R}}}1
         {Ž}{{\v{Z}}}1
         {Ď}{{\v{D}}}1
         {Ť}{{\v{T}}}1
         {Ň}{{\v{N}}}1                
         {Ů}{{\r{U}}}1    
}

\usepackage[Algoritmo]{algorithm}
\usepackage{algpseudocode}
\algrenewcommand\textproc{}% Used to be \textsc Para que el nombre de la función no sea en mayusculas
\usepackage{verbatim}

\usepackage{tabularx} % tablas copadas
% \usepackage{graphicx}
\usepackage{amsmath, amsthm, amssymb}

%\usepackage{makeidx}
\usepackage{paralist} %itemize inline
\usepackage[table]{xcolor}
\usepackage{caratula/caratula}

\usepackage{float}
\usepackage{amsfonts}
\usepackage{sectsty}
\usepackage{charter}
\usepackage{wrapfig}

\usepackage{caption}
%el de arriba funciona si no esta, el de abajo quite el "Figura #" de las imagenes
\captionsetup[figure]{labelformat=empty}

\begin{document}

\titulo{Trabajo práctico III}

\subtitulo{Sistemas Distribuidos}

\materia{Sistemas Operativos}

\integrante{Cámera, Joel Esteban}{257/14}{joel.e.camera@gmail.com}
\integrante{Lavia, Alejandro Norberto}{43/11}{lavia.alejandro@gmail.com}
\integrante{Guttman, Martín David}{686/14}{mdg\_92@yahoo.com.ar}


\maketitle

\tableofcontents

\newpage
\section{Introducción}

En el presente informe reseñamos las mejoras realizadas al servidor del juego de \textit{HaSObro} para que tener múltiples jugadores a la vez. Para esto, utilizamos la biblioteca \textbf{pthreads} de POSIX para realizar los cambios en el servidor de \textit{backend} así pueda atender a múltiples jugadores.

Como primera parte del mismo, exponemos la implementación de la estructura \textit{Read-Write lock}, donde se utilizan sólo variables de condición de la librería \textbf{pthreads} , y los respectivos test que demuestran que la estructura está libre de inanición y posée el comportamiento esperado. Como segunda parte, presentamos una implementación del servidor de \textit{backend multiusuario} con los detalles implementativos y las decisiónes de diseño del mismo que hacen que permita múltiples clientes jugando simultáneamente.


\end{document}
