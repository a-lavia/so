\section{Implementación \textit{Read-Write Lock}}

Aquí presentamos nuestra implementación del \textit{Read-Write Lock} donde exponemos su estructura y su funcionamiento. En ésta se podría generar el problema de inanición si se presentan los casos en donde haya siempre pedidos de lectura o escritura, al intentar realizar un pedido contrario a estos según el caso éste queda bloqueado. Por ello, también presentamos las medidas tomadas para evitar estos casos.


\subsection{Estructura}

Para poder implementar el \textit{Read-Write Lock} utilizamos una estructura basada en una variable de condición, dos mutex y un entero que representa la cantidad de lectores.

La estructura es la siguiente:

\begin{itemize}
	\item Mutex: \textbf{mtx}
	\item Mutex: \textbf{roomEmtpy}
	\item Variable de condición: \textbf{turn}
	\item Contador de lectores: \textbf{readers}
\end{itemize}


\subsection{Función \textit{rlock()}}

\begin{lstlisting}
Pido el mutex roomEmpty
Pido el mutex mtx

Aumento en uno el contador de la cantidad de lectores

Libero el mutex mtx
Libero el mutex roomEmpty
\end{lstlisting}

Lo primero realiza esta función es pedir los mutex roomEmpty y mtx. Una vez obtenidos estos, aumenta en uno la cantidad de lectores y luego, libera ambos mutex.

Se liberan ambos mutex para que otro lector o algún escritor que estén esperando para entrar en la sección critica puedan hacerlo.

\newpage

\subsection{Función \textit{wlock()}}

\begin{lstlisting}
Pido el mutex roomEmpty
Pido el mutex mtx

Mientras la cantidad de lectores es mayor que cero
	Espero que terminen de leer con la variable turn, sobre el mutex mtx


Libero el mutex mtx
\end{lstlisting}

Primero se piden los mutex roomEmpty y mtx.
Luego, mientras haya lectores, espera a que éstos terminen de leer. Esto se hace para garantizar el acceso exclusivo a la sección critica del escritor. Se realiza con un wait en la variable de condición \textbf{turn}, donde éste genera un signal a todos los procesos que esperan sobre este mutex menos el que realiza el wait. Esto es importante para cuando analicemos los runlock y wunlock.

Una vez que no haya nadie leyendo en la sección critica, se libera sólo el mutex mtx.
El mutex roomEmpty lo mantiene para garantizarse la exclusividad. Como cada proceso, cuando quiere leer o escribir, lo pide, quedarían todos esperando hasta que el proceso de escritura que lo tiene lo libere.

~
\subsection{Función \textit{runlock()}}

\begin{lstlisting}
Pedir mutex mtx

Disminuyo en uno el contador de la cantidad de lectores

Si la cantidad de lectores es igual a cero
	Dar turno al escritor que esté esperando

Liberar mutex mtx
\end{lstlisting}

Una vez terminada la lectura, se pide sólo el mutex mtx (ya que el mutex roomEmpty es para entrar a la sección critica). Se disminuye la cantidad de lectores y, si ésta llega a cero se realiza un signal a algún escritor, si lo hay.

Los lectores pueden entrar aquí cuando hay un escritor esperando porque, cuando el escritor hace un wait sobre la variable de condición y el mutex mtx, se realiza un signal a todos los threads que pidieron ese mutex menos el del escritor entonces los lectores no se quedarían esperando cuando piden el mutex aquí, sino que entran todos.

Por último, se libera el mutex mtx.

~
\subsection{Función \textit{wunlock()}}

\begin{lstlisting}
Se libera el mutex roomEmpty
\end{lstlisting}

Una vez terminada la escritura, se libera el mutex roomEmpty que se habia quedado el escritor para que no pueda entrar nadie mientras él esté en la sección critica.

Como el mutex roomEmpty se libera, ahora puede entrar cualquiera que lo haya pedido, tanto lector como escritor.


~
\subsection{Inanición}

Por lo expuesto anteriormente sobre el funcionamiento de nuestra estructura, ésta contiene los medios suficientes para evitar que una seguidilla de threads con una misma operación hagan que se bloquéen threads con la operación contraria.

En el caso de los lectores, estos entran pidiendo los mutex roomEmpty y mtx, aumentan la cantidad de lectores, y luego liberan ambos mutex, pudiendo hacer que sigan entrando lectores o algún escritor.

En el caso de los escritores, como necesitan ser los únicos en la sección critica, cuando entran piden ambos mutex pero sólo se quedan con roomEmpty, haciendo que no entre ningún otro proceso a esta sección, y liberan el mutex mtx para que los lectores que estaban en la sección critica terminen de leer. Luego, éste puede escribir en la sección critica ya que es el único dentro y, al salir, libera el mutex roomEmpty así puede entrar tanto lectores como escritores.

Si hay una seguidilla de lectores, y un escritor, éste toma el el mutex roomEmpty y hace que no entre más nadie. Espera a que terminen todos los lectores que estaban en la sección critica y luego procede a escribir. Una vez terminado libera el mutex roomEmpty y pueden entrar nuevamente tanto lectores como escritores.

Si hay una seguidilla de escritores, y un lector, éste puede entrar cuando se libere el mutex roomEmpty. Como el lector libera ambos mutex puede entrar cualquier otro tipo de proceso. Si es de lectura ocurre lo mismo. Si es de escritura, el lector tiene que esperar a que se libere el mutex mtx con el wait de la variable de condición, luego éste le lanza un signal al escritor para que pueda escribir y libera el mutex mtx.

Si entran de uno en uno, por ejemplo, primero el lector y luego un escritor, el lector pide los dos mutex, aumenta la cantidad de lectores y los libera. Luego entra el escritor, pide ambos mutex y se queda esperando hasta que termine el lector, liberando solo el mutex mtx para que éste pueda terminar. Cuando el lector termina la lectura, disminuye en uno la cantidad de lectores y, como es cero, lanza un signal al escritor para que pueda escribir. Luego, este escribe y cuando termina libera el mutex roomEmpty.

Si fuera al reves, primero un escritor y luego un lector. El escritor como toma el mutex roomEmpty y no lo libera hasta que haya terminado la lectura, el lector se queda esperando a que éste sea liberado. Cuando el escritor termina la escritura, libera el mutex roomEmpty y ahora puede entrar a la sección critica el lector.

~
\subsection{Deadlock}

Para mostrar que en la estructura no se genería deadlock veamos que no se cumplen las condiciones necesarias de Coffman.

\begin{itemize}
	\item \textbf{Exclusión mutua}: Un recurso no puede estar asignado a más de un proceso. 
\newline
	Cómo vimos, las operaciones de lock (tanto read como write) piden ambos mutex en el mismo orden, esto garantiza que el acceso a un recurso le pertenece a un sólo thread y todos los que quieran utilizarlo deben esperar hasta que se los libere.

	En el caso de la variable de condición en la función wlock, como es sobre el mutex mtx, este realiza un signal a todos los que lo pidieron menos al que espera la condición (en este caso el escritor). Entonces, este espera a que terminen de leer todos y, cuando leyo el último y sale de la sección critica, le envía un signal al escritor que estaba esperando y éste tiene en este momento tiene garantizado el acceso exclusivo a la sección.
	\item \textbf{Hold and wait}: Los procesos que ya tienen algún recurso pueden solicitar otro.
\newline
	En este caso, esto ocurriría si un thread no puede leer o escribir se quede con el mutex hasta que pueda realizar la operación. Pero esto no sucede ya que los lectores piden ambos mutex y los liberan y, en el caso de los escritores, estos piden los mutex, si hay escritores, éste libera a todos los procesos que hayan pedido el mutex mtx, por el llamado a  a la función \textbf{pthread\_cond\_wait()}, y luego, esperan a que hayan leído los lectores en la sección critica para que le envíen un signal y él poder escribir.
	
	\item \textbf{No Preemption}: Un recurso puede ser liberado solamente de forma voluntaria por el proceso que lo retiene.
\newline
	En este caso, no podría suceder esto ya que los threads que quieran acceder a un recurso que esta siendo utilizado deben esperar a que el mutex roomEmpty sea liberado.
	
	\item \textbf{Espera circular}: Tiene que haber un ciclo de $N \geq 2$, tal que $P_i$ espera un recurso que tiene $P_{i+1}$.
\newline
	Cómo el recurso al que se quiere acceder es único en este caso, no puede haber espera circular.
\end{itemize}


\subsection{Tests}

Para probar nuestra implementación de \textit{Read-Write Lock} realizamos tres tipos de test que se pueden encontrar en la carpeta \textbf{locks} junto con la implementación de la estructura.

Pasamos a explicar sobre ellos:

\begin{enumerate}
	\item En este primer test lanzamos 50 threads lectores al comienzo, despues un escritor y luego, 50 threads lectores más, en este orden. Cada uno de estos pide su lock (los de lectura el rlock y el de escritura el wlock) y cuando lo obtienen escriben \emph{Read} o \emph{Write} dependiendo que tipo de thread es. 
	Este test sirve para observar si los lectores tienen prioridad sobre el escritor. Si esto sucede, el thread de escritura que fue lanzado en el medio aparecería despues de que todos los threads de lectura hayan terminado.

	\item En el segundo test realizamos algo parecido al anterior, lanzamos 50 threads lectores, 10 threads escritores y 50 threads lectores.
	Con este test queremos observar si los lectores tienen prioridad (haciendo que los 10 escritores escriban después de todos los lectores); o si los escritores tienen prioridad (haciendo que cuando son lanzados, los lectores siempre esperen que todos los escritores terminen de escribir).

	\item En este tercer y último test lo que hacemos es lanzar 50 threads donde 25 son escritores y 25 son lectores. Éstos son lanzados de uno en uno. 
	Lo que queremos observar en éste test es si los escritores tienen prioridad sobre los lectores o viceversa. Éste test lo mostraria muy rápido ya que si alguno tiene prioridad sobre otro, este último esperaría a que los que tienen mayor prioridad terminen.
\end{enumerate}